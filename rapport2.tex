\documentclass[10pt,a4paper,danish]{article}
%% Indlæs ofte brugte pakker
\usepackage{amssymb}
\usepackage[danish]{babel}
\usepackage[utf8]{inputenc}
\usepackage{listings}
\usepackage{fancyhdr}
\usepackage{hyperref}
\usepackage{booktabs}
\usepackage{graphicx}
\usepackage{todonotes}
\usepackage{algorithmic}
\usepackage{amsmath}


\pagestyle{fancy}
\fancyhead{}
\fancyfoot{}
\rhead{\today}
\rfoot{\thepage}
\setlength{\parindent}{0pt}

% Opsæt indlæsning af filer
\lstset{
 language=Python,
 extendedchars=\true,
 inputencoding=utf8,
 linewidth=\textwidth, basicstyle=\small,
 numbers=left, numberstyle=\footnotesize,
 tabsize=2, showstringspaces=false,
 breaklines=true, breakatwhitespace=false,
}

%% Titel og forfatter
\title{G2\\Maskinarkitektur\\Efterår 2011}
\author{Jens Fredskov\\ Naja Mottelson\\Søren Pilgård}

%% Start dokumentet
\begin{document}

%% Vis titel
\maketitle
\newpage

%% Vis indholdsfortegnelse
\tableofcontents
\newpage

%% Klar, parat, start!
\section{Indledning}
Denne rapport dokumenterer gruppens arbejde med anden godkendelsesopgave i kurset Maskinarkitektur. Den indeholder, for såvidt vi kan se, korrekte besvarelser af begge underopgaver. 

\section{g2-1}
De instruktioner der gennemgåes grundigt i lærebogen (lw, sw, beq) har vi implementeret efter samme algoritme der præsenteres dér. Vi vil derfor ikke bruge yderligere tid på at gennemgå dem her. [NB til Naja: Denne sektion skal udvides en del. Det går nok ikke bare at sige 'dem snakker vi ikke om'.]

\section{addu, subu, and, or, nor, slt}
Alle R-instruktioner har vi implementeret på grundlæggende samme måde: I kontrollen sættes Write-registeradressen til rd (vha. RegDst-bitten) og RegWrite-bitten sættes så at det er muligt at skrive til registrene. Resten af kontrollogikken foretages i ALU-kontrollen som modtager en 3-bit wire indeholdende ALUop-signalet fra kontrollen samt de 6 bits der hentes ud fra operationens funct-felt. Ud fra dette input genererer ALU-kontrollen et output på 6 bits, hvoraf de fire mindst betydende sendes til ALU'en som signal om hvilken af de aritmetisk-logiske instruktioner den skal udføre, hvorefter outputtet fra ALU'en skrives til registerbanken. I tilfældet af de grundlæggende R-instruktioner benyttes de to sidste bits i ALU-kontrollens output ikke. 

\section{sll, srl, sra}
Da skifteoperationerne ligeledes er R-instruktioner er de implementeret med samme kontrollogik som ovenfor. Her benyttes den fjerde bit i ALU-kontrollens output dog til at give shamt som den første operand til ALU'en i stedet for register rs. 

\section{addiu, andi, ori, slti} 
I-instruktionerne er implementeret på samme facon som R-instruktionerne med de undtagelser at der skrives til register rt samt at ALUsrc-bitten sættes i kontrollen. ALUsrc gives som selector-input til en MUX som vælger hvilket input der gives til ALU'ens anden operand. Når ALUsrc er sat vælges operationens immediate-felt, som er blevet enten nul- eller fortegnsudvidet afhængigt af instruktionen. Nul- eller fortegnsudvidelse specificeres ved kontrollens SignExtend-bit. 

\section{j, jal}
J-instruktionerne 



\section{g2-2}

\end{document}

